
\section{Introdução}




\noindent \begin{minipage}[c]{0.6\textwidth}
  \vspace {1cm}
\par Na diciplina de programação e desenvolvimento de banco de dados é apresentado as discentes a linguagem \textbf{SQL} \textit{Structured Query Language}\ref{fig:logo}, e para a manipulação desta linguagem é proposto o software \textbf{MySQL da Workbench}

\end{minipage}
\begin{minipage}[c]{0.4\textwidth}
  \captionof{figure}{Logo SQL}
  \includegraphics[width=\textwidth]{figure/logo_sql.png}
  	\label{fig:logo}
    {\fontsize{10pt}{\baselineskip}\selectfont
    Fonte: \citeonline{logo:2023}
  }

\end{minipage}

\par Na figura \ref{fig:div_sql}, expõe a divisão da linguagem \textbf{SQL}, a masma é dividida em cinco suu conjuntos, endo eles: \textbf{DQL \label{DQL}}, \textbf{DML}, \textbf{DDL \label{DDL}}, \textbf{DCL} e \textbf{DTL}, cada uma com suas respectivas funções. por exemplo a \textbf{DQL \ref{DQL}}, é a linguagem de consulta de dados, definida pelo comando \textit{SELECT}, ao qual possibilita a consulta do dados armazenados no banco de dados.

\begin{figure}[h!]
  \caption{Sub divisões da Linguagem SQL}
  \includegraphics[width=\textwidth]{figure/div_SQL.png}
  \label{fig:div_sql}
  {\fontsize{10pt}{\baselineskip}\selectfont
  Fonte: \citeonline{guia:2023}  }
\end{figure}
\par Para esta aula prática é proposto o uso da \textbf{DDL}, Linguagem de definição de dados, a qual define os comandos \textit{CREATE}, \textit{ALTER} e \textit{DROP}, sendo elas na sequância, Criação de tabelas, visualzaões e índices; Alteração das estruturas e a remoção das estruturas criadas.

\section{Desenvolvimento}
\par Para implementação desta aula prática formam estabelecidos algumas regras informadas no roteiro \href{https://github.com/OgliariNatan/database_and_data_development/blob/main/aula%20pr%C3%A1tica.pdf}{da aula prática}. sendo a atividade proposta:
\begin{itemize}
  \item Criar uma estrutura de um banco de dados com a linguagem \textbf{SQL} por meio de uma entidade-relacionamento pré-definido;
  \item Inserir dados no banco de dados criado;
  \item Consultar os dados armazenados por meio da criação de uma vião \textit{View};
  \item Criar um relatório no final da atividade;
\end{itemize}
\par Na atividade proposta o relatório dispõe de alguns procedimentos para a realização da atividade. Sugere a criação de uma base de dados de uma loja com o nome do banco de \textbf{Loja}, com a utilizazão de definições de dados \textbf{DDL  \textsubscript{\ref{DDL}}} da linguagem SQL, e respeítando o modelo definido no \textbf{DER}, porposto pela atividade conforme figura \ref{fig:DER}.

\begin{figure}[h!]
  \caption{Diagrama entidade relacionamento}
  \includegraphics[width=\textwidth]{figure/diagram_EER.png}
  \label{fig:DER}
  {\fontsize{10pt}{\baselineskip}\selectfont
  Fonte: O autor (2023).}
\end{figure}

\newpage

\section{Método}

\lstinputlisting[caption={inserir.sql}, label={cod:inserir}]{inserir.sql}



\section{Conclusões}


  %$X \xLongleftarrow[\text{NATAN}]{\text{OGLIARI}} Y $ %COM TEXTO
	% $\uparrow$ %Seta para Cima
	%$\overleftarrow{NATAN}$
